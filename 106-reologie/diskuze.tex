\section*{Diskuze}
Na láhvi, ze které jsme brali glycerin, byla nálepka, že není zaručeno, že to je čistý glycerin.

Zdroj \cite{skripta} uvádí pro glycerin následující viskozity


\begin{tabular}[h]{ccc}
$c$~\textbackslash~$t$ & \SI{20}{\degreeCelsius} & \SI{30}{\degreeCelsius} \\ \hline
\SI{97}{\percent} & \SI{765}{\milli\pascal\s} & \SI{340}{\milli\pascal\s} \\
\SI{98}{\percent} & \SI{939}{\milli\pascal\s} & \SI{409}{\milli\pascal\s} \\
\end{tabular}

Je vidět, že viskozita glycerinu je velmi silně závislá na koncentraci i na teplotě.
Naše naměřená viskozita je v rozmezí vytyčeném těmito čtyřmi hodnotami.

Hodnoty viskozity naměřené rotačním viskozimetrem měly značný rozptyl, např. glycerin s rotorem \emph{L3} při \num{30} otáčkách za minutu jsme změřili čtyřikrát, hodnoty v \si{\milli\pascal\s} byly postupně 630, 840, 820, 760.
V takovém případě jsme vzali hodnotu 760 jako přibližný střed.
Tento jev byl pravděpodobně způsoben chybou přístroje.