\section*{Diskuze}
Podle \cite{converter} je při \SI{25}{\degreeCelsius} rychlost zvuku v oxidu uhličitém \SI{259}{\m\per\s}.
Naše naměřená hodnota \SI{269.1(2)}{\m\per\s} se od ní mírně liší.
Tato odchylka může být způsobena pouze teplotním rozdílem (teplota při našem měření byla \SI{26.1(4)}{\degreeCelsius}), v tom případě by bylo vše v pořádku.
Je ale také možné, že se nám nepodařilo rezonátor dostatečně "profouknout" a oxid uhličitý v něm byl naředěný vzduchem.

Podle \cite{poisson} je Poissonova konstanta oxidu uhličitého při \SI{20}{\degreeCelsius} rovna \num{1.40}. Naše naměřená hodnota $\kappa =~\num{1.280(3)}$ se s ní příliš neshoduje.

Podle \cite{sengpiel} je při teplotě \SI{26}{\degreeCelsius} a relativní vlhkosti \SI{36}{\percent} rychlost zvuku ve vzduchu přibližně \SI{347.6}{\m\per\s}.
Zdroj \cite{converter} uvádí při \SI{20}{\degreeCelsius} rychlost \SI{343}{\m\per\s} a při \SI{30}{\degreeCelsius} rychlost \SI{349}{\m\per\s}.
Naši naměřenou hodnotu \SI{345,7(4)}{\m\per\s} považujeme za přibližně správnou.

Podle \cite{converter} je rychlost zvuku v mosazi přibližně \SI{3400}{\m\per\s}. Naše naměřená hodnota \SI{3404(42)}{\m\per\s} se s ní velmi přesně shoduje.

Podle \cite{modul} je modul pružnosti v tahu mosazi \SI{99}{\GPa}. Naše naměřená hodnota \SI{99(3)}{\GPa} se s ní shoduje.