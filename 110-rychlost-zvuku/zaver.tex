\section*{Závěr}
Pomocí uzavřeného rezonátoru jsme změřili rychlost zvuku ve vzduchu dvěma metodami, při konstantní délce rezonátoru jsme měnili frekvenci zdroje a poté jsme při konstantní frekvenci zdroje měřili délky rezonátoru, při kterých nastala rezonance.
Rychlost zvuku ve vzduchu vyšla shodně oběma metodami \SI{345.7(4)}{\m\per\s}.

První metodou (při konstantní délce rezonátoru) jsme také měřili rychlost zvuku v oxidu uhličitém \linebreak \SI{269.1(2)}{\m\per\s}.
Pomocí této změřené rychlosti zvuku jsme spočítali Poissonovu konstantu oxidu uhličitého $\kappa =~\num{1.280(3)}$.

Dále jsme pomocí Kundtovy trubice změřili rychlost zvuku v mosazné tyči \SI{3404(42)}{\m\per\s}.
V závislosti na rychlosti zvuku a hustotě jsme vypočítali modul pružnosti v tahu $E$ mosazi \SI{99(3)}{\GPa}.


Všechny uvedené odchylky jsou standardní ($P=\num{0.68}$).