\section*{Diskuze}
Kruhovou frekvenci netlumeného kmitání $\omega$ jsme měřili bez tlumícího disku a k magnetu jsme připevnili přívažek, který měl lehce odlišnou hmotnost než disk.
Tato naměřená hodnota $\omega$ se tedy může od skutečné mírně lišit, což by mělo za následek nepřesnost při výpočtu $\omega_1$ a $\Omega_{res}$.

Teoretická hodnota podle \eqref{eq::omega1} vychází díky velmi slabému tlumení totožně pro všechny disky \SI{6.89}{\radian\per\s}.
Skutečné hodnoty jsou nižší.
Naměřené kruhové frekvence se výrazně liší i pro různé disky a odpovídají teorii v tom smyslu, že s rostoucí konstantou tlumení klesá.
Aby však měly za následek takové rozdíly naměřených $\omega_1$ mezi různými tlumícími disky, musely by být téměř o dva řády vyšší.
To znamená, že se nám buď nepodařilo správně změřit konstantu tlumení, nebo vztah \eqref{eq::omega1} v našem pokusu neplatí.

Z grafu \ref{grp::modrytlum} je také vidět, že amplituda tlumených kmitů neměla zcela exponenciální průběh, hodnoty ve středu jsou více pod křivkou, zatímco hodnoty na krajích jsou většinou nad křivkou.
Usuzujeme, že tlumící síla nebyla přesně přímo úměrná rychlosti pohybu, ale při vyšších rychlostech byla větší.
Pokud bychom závislost nefitovali čistou exponenciálou, ale přidali bychom absolutní člen, vyšla by konstanta tlumení pro modrý disk přibližně \SI{0.055}{\per\s}, tento postup ale zřejmě nemá fyzikální význam.

Fázové posunutí nám pro vyšší frekvence vyšlo větší než \SI{180}{\degree}, což podle \cite{skripta} není možné.
Tuto skutečnost považujeme za chybu přístroje nebo metody.