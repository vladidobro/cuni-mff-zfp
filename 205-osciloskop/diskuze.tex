\section*{Diskuze}
V úkolu 1. se hodnoty naměřené osciloskopem i voltmetrem dobře shodují.

Závislost stejnosměrného napětí na filtrační kapacitě vyšla podle očekávání rostoucí a pro velké kapacity blížící se špičkové hodnotě.
Časový průběh napětí se na první pohled shodoval s předpokládaným průbehem (viz \cite{skripta}).

Hodnoty $\tau$ pro jednotlivé proudy se sice trochu liší, ale rozdíly jsou malé a jsou způsobeny malým rozlišením osciloskopu, která se projevila nepřesností při určování kapacity.

Závislost kapacity na proudu zátěží (graf \ref{g:druhy}) vyšla podle očekávání lineární.

Činitel filtrace byl $k_f = \num{6.9(2)} \gg 1$, tedy použité aproximace byly opodstatněné.