\section*{Teoretická část}
Pro střídavé napětí harmonického průběhu platí vztah mezi špičkovým napětím $U_0$ a efektivním napětím $U_{ef}$
\begin{equation} \label{e:efektivni}
U_{ef}=\frac{U_0}{\sqrt{2}} \,.
\end{equation}

Budeme měřit jednocestný usměrňovač. \cite{skripta}
Střední hodnota $U_e$ jednocestně usměrněného napětí je \cite{skripta}
\begin{equation}
U_e=\frac{U_0}{\pi} \,.
\end{equation}

Při zapojení pro úkol 2. (b) by napětí mezi jednotlivými pulzy mělo mít přibližný průběh \cite{skripta}
\begin{equation}
u \approxeq U_0 \left( 1 - \frac{t}{\tau}  \right) \,,
\end{equation}
kde
\begin{equation}
\tau = R_z C 
\end{equation}
je časová konstanta vybíjení.

Definujeme činitel filtrace
\begin{equation}
k_f = \frac{U_0}{\Delta U} \approxeq \frac{\tau}{T} \,,
\end{equation}
kde $T$ je doba mezi pulzy.

Pro kapacitu $C$ platí vztah \cite{skripta}
\begin{equation}
C=\frac{T k_f I_{SS}}{U_0} \,,
\end{equation}
kde $I_{SS}$ je proud zátěží.

Pokud tedy budeme udržovat činitel filtrace konstantí, měla by kapacita být přímo úměrná proudu $I_{SS}$, a $\tau$ by mělo být konstantní.