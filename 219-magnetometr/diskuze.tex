\section*{Diskuze}
Z Biotova-Savartova zákona plyne, že bychom pro jeden konkrétní magnet měli být schopni najít konstantu $p$ tak, že naměřené hodnoty pro obě cívky a všechny proudy budou odpovídat vzorci \eqref{eq:vzorecalfa}.
Takovou konstantu se nám skutečně podařilo najít, naměřené hodnoty vykazují dobrou shodu s teoretickou závislostí (viz graf \ref{graf:grafmain}).
Můžeme soudit, že naše výsledky jsou ve shodě s Biotovým-Savartovým zákonem.

Domnělým zdrojem velkých nepřesností byly otřesy v místnosti, avšak i po důkladném dupání v bezprostřední blízkosti aparatury nebyly naměřeny žádné odchylky.
Další chyby mohly být způsobeny nedokonalým tvarem cívek, nelinearitou vlákna či kolísáním proudu, tyto chyby však považujeme za malé a měření za neobyčejně přesné.

Naopak měření direkčního momentu vlákna považujeme za nepříliš přesné, k měření délky mosazné tyče byl k dispozici pouze svinovací metr a ostatní parametry tyče byly napsané na přiloženém papírku bez údaje o jejich přesnosti.
Kromě toho jsme zanedbávali ostatní části aparatury zavěšené na vlákně.
Ve vzorci \eqref{eq:momentsetrvacnosti} jsme mohli zanedbat první člen.

