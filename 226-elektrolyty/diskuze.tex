\section*{Diskuze}

Z grafu \ref{g:sCl} vyplývá, že závislost konduktivity \ce{HCl} na koncentraci je velmi dobře lineární a z grafu \ref{g:lCl} nevyplývá žádná jednoduchá závislost molární konduktivity na koncentraci.
Závislost není dokonce ani monotónní, což neodpovídá empirickému vzorci \eqref{e:silne}.
První čtyři hodnoty naznačují klesající trend, pokud bychom proložili přímkou jen je (v grafu přerušovanou čarou), obdrželi bychom mírně odlišnou hodnotu $\Lambda_0=\SI{38.4(2)}{\milli\siemens\metre\squared\per\mole}$. Tento postup ale považujeme za neoprávněný.
Konstanta $k$ je pravděpodobně příliš malá na to, abychom ji změřili našimi prostředky.

\ce{CH_3COOH} není silný elektrolyt a proto pro ni neplatí \eqref{e:silne}. Z grafu \ref{g:lCH} je vidět, že závislost $\Lambda(\sqrt{c_M})$ v měřeném rozsahu určitě není lineární.
Pokud bychom předpokládali, že se na intervalu $\left(0;\num{0.7}\right)$ chová lineárně, extrapolací bychom získali $\Lambda_0=\SI{17.8(5)}{\milli\siemens\metre\squared\per\mole}$. Odpovídající závislost $\sigma(c_M)$ je zakreslena do grafu \ref{g:sCH}. Ve skutečnosti ale graf \ref{g:lCH} naznačuje spíše ryze konvexní průběh, takže můžeme usuzovat, že skutečná hodnota je vyšší.
Uvedené úvahy jsou ovšem pouhé dohady, pro spolehlivé určení $\Lambda_0$ pro \ce{CH_3COOH} by bylo nutné důkladnější měření a studium slabých elektrolytů.


Zjistili jsme, že molární konduktivita silného elektrolytu je přibližně čtyřikrát vyšší než slabého elektrolytu.
Se zvyšující se koncentrací u slabého elektrolytu velmi rychle klesá, zatímco u silného elektrolytu klesá velmi pomalu, či je dokonce konstantní.