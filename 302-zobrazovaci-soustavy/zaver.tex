\section*{Závěr}
Změřili jsme ohniskovou vzdálenost tenké čočky Besselovou metodou
\begin{equation*}
f_B=\SI{10.82(6)}{\cm}
\end{equation*}
a metodou dvojího zvětšení
\begin{equation*}
f_{a} = \SI{10.2(10)}{\cm} \qquad \qquad f_{\apr}=\SI{10.4(1)}{\cm} \,.
\end{equation*}

Změřili jsme kulovou vadu téže čočky pro dvě různé předmětové vzdálenosti, viz tabulka \ref{t:kulvada} a graf \ref{g:vada}.

Změřili jsme vzdálenost hlavních rovin tenké a tlusté čočky
\begin{equation*}
\delta_{tenka} = \SI{12.9(3)}{\mm} \qquad \qquad \delta_{tlusta} = \SI{3.9(3)}{\mm}
\end{equation*}

Na fokometru jsme změřili optickou mohutnost tenké čočky v obou směrech
\begin{equation*}
\varphi_V=\SI{10.25(25)}{\per\metre} \qquad \qquad \varphi_P=\SI{9.75(25)}{\per\metre} \,.
\end{equation*}

Ze známé tloušťky čočky a vzdáleností jejích hlavních rovin jsme určili index lomu skla, ze kterého byla zhotovena
\begin{equation*}
n_{tlusta}=\num{1.51(3)} \,.
\end{equation*}