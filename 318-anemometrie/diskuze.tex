\section*{Diskuze}
Obě metody kalibrace "optické sondy anemometru" daly přibližně stejný výsledek. Geometrickou metodu považujeme za poněkud přesnější. Při projekční metodě vznikla chyba především při určování, zda je proužek již zakryt, nebo ne.


Běhěm samotného měření se nepodařilo odizolovat periodický šum, který byl mnohokrát silnější než měřený signál. Pravděpodobně šlo o \SI{50}{\Hz} šum ze sítě. Tento šum do značné míry znemožnil měření. Byli jsme nuceni měřit i částice, které bychom za normálních okolností neměřili.


Histogram na grafu \ref{g:hist} příliš nepřipomíná normální rozdělení. Vinu připisujeme zmíněnému šumu a jím způsobeným problémům. 


Voda má index lomu rozdílný od vzduchu, takže při přechodu se změní vlnová délka $\lambda$ i úhel $\vartheta$. Ve vzorci \eqref{eq:kal_inter} ale tyto dvě změny působí proti sobě
\begin{equation*}
d_F=\frac{\lambda\prime}{2} \frac{1}{\sin\frac{\vartheta\prime}{2}}=\frac{\lambda}{2n} \frac{n}{sin\frac{\vartheta}{2}} \,,
\end{equation*}
kde $n$ je index lomu vody a čárkované veličiny značí hodnoty ve vodě. Skutečnost, že jsme "optickou sondu anemometru" kalibrovali ve vzduchu, tedy na výsledek nemá vliv.