\documentclass[a4paper]{article}

\usepackage[czech]{babel} %https://github.com/michal-h21/biblatex-iso690
\usepackage[
   backend=biber      % if we want unicode 
  ,style=iso-numeric % or iso-numeric for numeric citation method          
  ,babel=other        % to support multiple languages in bibliography
  ,sortlocale=cs_CZ   % locale of main language, it is for sorting
  ,bibencoding=UTF8   % this is necessary only if bibliography file is in different encoding than main document
]{biblatex}

\usepackage[utf8]{inputenc}
\usepackage{fancyhdr}
\usepackage{amsmath}
\usepackage{amssymb}
\usepackage[left=2cm,right=2cm,top=2.5cm,bottom=2.5cm]{geometry}
\usepackage{graphicx}
\usepackage{pdfpages}
\usepackage{url}
\usepackage{multirow}
\usepackage{siunitx}
\sisetup{locale = DE, separate-uncertainty = true} %    kdybych chtel +/-
\usepackage{caption}
\usepackage{subcaption}
\usepackage{float}
\newfloat{graph}{htbp}{grp}
\floatname{graph}{Graf}
\newfloat{tabulka}{htbp}{tbl}
\floatname{tabulka}{Tabulka}

\renewcommand{\thefootnote}{\roman{footnote}}

\pagestyle{fancy}
\lhead{Praktikum III - (21) Koherentní optický procesor}
\rhead{Vladislav Wohlrath}
\author{Vladislav Wohlrath}

\bibliography{source}

\begin{document}

\begin{titlepage}
\includepdf[pages={1}]{./graficos/titlelist.pdf}
\end{titlepage}

\section*{Pracovní úkoly}
\begin{enumerate}
\item Na optickém stole je sestaven Koherentní optický procesor. Na obr. v „Pokynech měření“ (nebo skripta str. 188) je vyznačeno schematické uspořádání a vyznačeny ohniskové délky čoček, které jsou v úloze k dispozici. Ověřte, zda čočky Č1 a Č2 zachovávají rovnoběžnost paprsků. Spočtěte a ověřte rozšíření paprsku použitým teleskopem. Změřte zvětšení obrazu předmětu v rovině P3 a zvětšení obrazu Fourierova spektra v rovině P4.
\item Pozorujte Fourierovský obraz následujících tří předmětů umístěných v rovině P1: čtvercového otvoru, soustavy rovnoběžných pruhů a síťky. Proměřte Fourierova spektra těchto předmětů v rovině P2 nebo P4 a z naměřených údajů vypočítejte rozměry předmětů, tj. velikost stran čtvercového otvoru, šířku a periodu soustavy rovnoběžných pruhů a periodu a šířku pruhů síťky.
\item Parametry předmětů z úkolu 2 změřte mikroskopem, který je v úloze č. 6, č. 30 nebo č. 14. Porovnejte hodnoty vypočtené z Fourierova spektra s přímým měřením mikroskopem.
\item Po dohodě s vyučujícím vyberte a kvalitativně ověřte některou z vlastností Fourierovy transformace, které jsou uvedeny v odd. 4.10.2 části I skript nebo na www.
\item V rovině P1 umístěte vybraný předmět. Do roviny P2 vkládejte různé filtry a zkoumejte jejich vliv na geometrický obraz v rovině P3. Pozorované jevy vysvětlete.

\end{enumerate}

%Teoretická část
\section*{Teoretická část}
Fourierovu transformaci definujeme jako v \cite{cerny}.


Čočka vytváří ve své obrazové předmětové rovině rozložení amplitudy světla úměrné Fourierově transformaci rozložení amplitudy světla v předmětové rovině. \cite{skripta}
Používáme aparaturu jako v \cite{skripta}


V následujícím budeme $\lambda$ značit vlnovou délku laseru, $f$ ohniskovou vzdálenost čočky Č a $x$, $y$ budeme značit souřadnice ve Fourierově rovině s počátkem v optické ose. Pokud do předmětové roviny umístíme obdélník se stranami $a$ resp. $b$ ve směru $x$ resp. $y$, budou ve Fourierově rovině minima intenzity v bodech
$x = k\lambda f /a$ nebo $y=k\lambda f/b$ pro celé $k$ různé od nuly. \cite{maly}

Pokud do předmětové roviny umístíme soustavu svislých štěrbin širokých $b$, od sebe vzdálených $l$, budou ve Fourierově rovině maxima intenzity v bodech $x=k\lambda f/l$, $y=0$ pro celé $k$. Intenzita bude ve vodorovném směru modulována funkcí sinc s minimy v bodech $x=k\lambda f/d$ pro celé $k$ různé od nuly. \cite{maly}

Pokud do předmětové roviny umístíme síť čtvercových otvorů se stranou $c$ od sebe vzdálených $w$, budou ve Fourierově rovině maxima intenzity v bodech $x=m\lambda f/w$, $y=n\lambda f/w$ pro celé $m$, $n$. Intenzita bude opět modulovaná funkcemi sinc s minimy v bodech $x=k\lambda f/c$ nebo $y=k\lambda f/c$ pro celé $k$ různé od nuly.

%Výsledky měření
\section*{Výsledky měření}
Teplota v místnosti byla $T_0 = \SI{24.0(3)}{\degreeCelsius}$.

Teplotu jsme měřili platinovým odporovým teploměrem.
Vztah mezi teplotou $t$ ve stupních Celsia a odporem platinového teploměru $R_t$ je
\begin{equation}
t=\frac{R_t - R_0}{\alpha_{Pt} R_0} \,,
\end{equation}
kde $R_0 = \SI{100}{\ohm}$ je odpor při \SI{0}{\degreeCelsius} a $\alpha_{Pt} = \SI{3.85e-3}{\per\kelvin}$.

Elektrické veličiny (napětí, proud, odpor) jsme měřili multimetrem METEX MXD-4660A.
Při měření statické charakteristiky jsme hodnoty odečítali přímo z displeje a zapisovali na papír, při měření tepelné závislosti jsme multimetry připojili k počítači.



Naměřená teplotní závislost je uvedena v přiložené tabulce 1 a zanesena v grafu \ref{g:log}.
Proložením závislosti jsme určili konstanty $B =\SI{2680(30)}{\kelvin}$ a $R_\infty = \SI{7.0(1)}{\milli\ohm}$.
Chybu těchto veličin jsme odhadli, přičemž jsme uvážili chybu fitu a chyby fitovaných veličin.

\begin{graph}[htbp] 
\centering
\input{logplot.tex}
\caption{Teplotní závislost odporu}
\label{g:log}
\end{graph}

Ze známé konstanty $B$ a vztahů \eqref{e:aktivacni}, \eqref{e:Unamol}, \eqref{e:Tm} a \eqref{e:soucinitel} získáme
$\Delta U= \SI{0.463(6)}{\electronvolt}$,
$\Delta U_{mol}=\SI{44.6(5)}{\kilo\joule\per\mole}$,
$T_m=\SI{340(2)}{\kelvin}$ a
$\alpha=\SI{-0.0304(5)}{\per\kelvin}$
(při pokojové teplotě \SI{24.0}{\degreeCelsius}).
Odchylku těchto veličin jsme počítali metodou přenosu chyb.\cite{prakt}


Naměřená statická charakteristika je uvedena v přiložené tabulce 2
a zanesena do grafu \ref{g:stat}.

\begin{graph}[htbp] 
\centering
\input{statplot.tex}
\caption{Statická charakteristika}
\label{g:stat}
\end{graph}

Z grafu jsme určili $U_m = \SI{1.086(5)}{\volt}$ a $I_m=\SI{6.0(3)}{\milli\ampere}$.
Podle \eqref{e:tepodpor} dostáváme $K=\SI{6600(400)}{\kelvin\per\watt}$.

%Diskuze výsledků
\section*{Diskuze}


Funkci $f(T)$ jsme určili až na škálovací faktor způsobený nepřesným $r$. Přesto závislost přibližně odpovídá teoretické závislosti \eqref{teoreticka}, což napovídá, že jsme vzdálenost $r$ odhadli správně.

V numerické derivaci \eqref{dTdP} odčítáme ve jmenovateli blízká čísla, což způsobuje chybu, pokud jsme tlak nezměřili přesně.

Tlak vzduchu byl $P_0=\SI{960}{\hecto\pascal}$, což bylo způsobeno jinou teplotou a vlhkostí než je normální ($P_0=\SI{1013}{\hecto\pascal}$). Proto jsme i specifické ionizační ztráty $f$ vztahovali ve vzorci \ref{f} k tlaku \SI{960}{\hecto\pascal} a dostali jsme funkci, jejíž teoretický tvar se může lišit od \eqref{teoreticka}.

Píky \Pudev a \Puosm se poměrně dobře shodují s tabelovanými hodnotami, jsou ale nižší, což bylo pravděpodobně způsobeno nedokonalým vakuem nebo kalibrací.

Ke kalibraci jsme použili pouze dva body (jeden pík a nulový kanál) a omezili jsme se tedy jen na lineární závislost čísla kanálu a energie. Nejsme tedy schopni posoudit kvalitu kalibrace.

%Závěr
\section*{Závěr}
Zpracovali jsme 106 událostí.

Na histogramech jsme rozpoznali boson Z a H, dále simulované Z' a g, a pravděpodobně také J/$\uppsi$ a $\Upsilon$, viz grafy \ref{o:m1} až \ref{o:a4}.

Zobrazili jsme si histogramy pro různě velké statistické soubory. Podle očekávání se parametry fitované Gaussovy funkce příliš neměnili, pouze se snižovala nejistota jejich určení a to úměrně $1/\sqrt{N}$, jak vyplývá z Poissonova rozdělení.


\printbibliography[title={Seznam použité literatury}]

\end{document}