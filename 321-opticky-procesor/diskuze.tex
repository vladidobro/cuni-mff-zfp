\section*{Diskuze}
Změřené rozšíření svazku se shoduje s teoretickou hodnotou v rámci standardní odchylky. Měření průměru svazku před dalekohledem bylo velice nepřesné.

Rozměry předmětů se sice poměrně dobře shodují s těmi naměřenými mikroskopem, nicméně v rámci standardní odchylky se neshodují (u čtverečku dokonce ani $3\sigma$). To nasvědčuje tomu, že je buď podhodnocená chyba měření vzdáleností maxim a minim intenzity, chyba $f$ není nezanedbatelná, nebo se uplatňuje ještě nějaká jiná neodhalená systematická chyba. Hodnoty změřené z Fourierova spektra jsou vždy přibližně o \SI{8}{\percent} nižší, pravděpodobně jde tedy o systematickou chybu.
Graf \ref{g:ctv} naznačuje, že při měření ve svislém směru mohlo dojít k hrubé chybě při počítání řádů minim intenzity.

Dolní odhad pro šířku štěrbin změřený z Fourierova spektra je skutečně menší než hodnota naměřená na mikroskopu.