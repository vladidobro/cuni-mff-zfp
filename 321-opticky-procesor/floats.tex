%PRUHY

\begin{tabulka}[htbp]
\centering
\begin{tabular}{c|cccccccc}
řád maxima  & -3 & -2 & -1 & 0 & 1 & 2 & 3 & 4 \\ \hline
relativní poloha (\si{\mm})  & \num{-19.62} & \num{-13.84} & \num{-7.79} & \num{0.00} & \num{6.00} & \num{10.66} & \num{15.80} & \num{21.27} \\
\end{tabular}
\caption{Relativní polohy maxim intenzity ve Fourierově obrazu soustavy průhů}
\label{t:pruhy}
\end{tabulka}

\begin{tabulka}[htbp]
\centering
\begin{tabular}{c|ccccc||ccccc}
& \multicolumn{5}{c||}{řád maxima} & \multicolumn{5}{c}{řád minima\footnotemark} \\
 &  -3 & -1 & 0 & 1 & 3  & -2 & -1 & 0 & 1 & 2 \\ \hline
vodorovně (\si{\cm})  & \num{-1.7} & \num{-0.6} & \num{0.0} & \num{0.5} & \num{1.6} & \num{-2.4} & \num{-1.3} & \num{0.0} & \num{1.2} & \num{2.2} \\
svisle (\si{\cm})  & \num{-1.4} & \num{-0.2} & \num{0.4} & \num{1.0} & \num{2.1} & \num{-0.4} & \num{0.8} & \num{2.0} & \num{3.2} & \num{4.1} \\
\end{tabular}
\caption{Relativní polohy maxim a minim intenzity ve Fourierově obrazu síťky}
\label{t:sitka}
\end{tabulka}
\footnotetext{Nultý řád ve skutečnosti není minimum, přesto jsme ho změřili a uvádíme kvůli zvýšení počtu fitovaných hodnot.}