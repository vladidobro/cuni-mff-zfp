\section*{Teoretická část}
Fourierovu transformaci definujeme jako v \cite{cerny}.


Čočka vytváří ve své obrazové předmětové rovině rozložení amplitudy světla úměrné Fourierově transformaci rozložení amplitudy světla v předmětové rovině. \cite{skripta}
Používáme aparaturu jako v \cite{skripta}


V následujícím budeme $\lambda$ značit vlnovou délku laseru, $f$ ohniskovou vzdálenost čočky Č a $x$, $y$ budeme značit souřadnice ve Fourierově rovině s počátkem v optické ose. Pokud do předmětové roviny umístíme obdélník se stranami $a$ resp. $b$ ve směru $x$ resp. $y$, budou ve Fourierově rovině minima intenzity v bodech
$x = k\lambda f /a$ nebo $y=k\lambda f/b$ pro celé $k$ různé od nuly. \cite{maly}

Pokud do předmětové roviny umístíme soustavu svislých štěrbin širokých $b$, od sebe vzdálených $l$, budou ve Fourierově rovině maxima intenzity v bodech $x=k\lambda f/l$, $y=0$ pro celé $k$. Intenzita bude ve vodorovném směru modulována funkcí sinc s minimy v bodech $x=k\lambda f/d$ pro celé $k$ různé od nuly. \cite{maly}

Pokud do předmětové roviny umístíme síť čtvercových otvorů se stranou $c$ od sebe vzdálených $w$, budou ve Fourierově rovině maxima intenzity v bodech $x=m\lambda f/w$, $y=n\lambda f/w$ pro celé $m$, $n$. Intenzita bude opět modulovaná funkcemi sinc s minimy v bodech $x=k\lambda f/c$ nebo $y=k\lambda f/c$ pro celé $k$ různé od nuly.