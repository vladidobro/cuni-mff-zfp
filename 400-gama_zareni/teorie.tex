\section*{Teoretická část}
Budeme studovat spektrum gama záření polovodičovým spektrometrem. Po dopadu fotonu vznikají elektrony, jejichž energii detektor měří. Výsledné spektrum je energetické spektrum takto vzniklých elektronů.
Elektrony vznikají třemi hlavními způsoby.
\begin{enumerate}
\item \emph{Fotoefekt}\cite{skripta} |
Foton vyrazí z atomu elektron a předá mu všechnu svoji energii. Výstupní práce je zpravidla zanedbatelná vůči energii fotonu, takže pík bude mít energii téměř rovnou energii vstupního gama záření. Mluvíme o \emph{píku plné absorbce} (FEP).
\item \emph{Comptonův jev}\cite{skripta} | Foton se pružně rozptýlí na volném elektronu a předá mu část své energie. Největší možnou energii $E_c$ foton předá při rozptylu o úhel \SI{180}{\degreeCelsius}. Pokud tento rozptýlený foton z detektoru unikne, je spektrum spojité v energiích od 0 do $E_c$. V $T_c$ tedy budeme ve spektru pozorovat \emph{comptonovu hranu}.

Rozptýlený foton však může dále interagovat. Pokud se například ještě jednou pružně rozptýlí a poté unikne, může předat maximální energii $E_{2c}>E_c$ a budeme pozorovat spojité spektrum od 0 do $E_{2c}$. V $E_{2c}$ budeme tedy pozorovat další hranu.
\item \emph{Tvorba elektron-pozitronových párů}\cite{skripta} | Pokud je energie dopadajícího fotonu dostatečně vysok (větší než $2m_ec^2=2\cdot\kev{511}$), může foton zaniknout a vytvořit pár elektron-pozitron. Pozitron velmi rychle anihiluje za vzniku dvou fotonů s energií \kev{511} a opačnou hybností. Fotony dále interagují s detektorem a pozorujeme podobné efekty jako výše.

Pokud jeden z fotonů unikne z detektoru, naměříme energii o \kev{511} menší než energii dopadajícího fotonu, pak mluvíme o \emph{píku jednoho úniku} (SEP). Pokud uniknou oba fotony, naměříme o $2\cdot \kev{511}$ menší energii a mluvíme o \emph{píku dvou úniků} (DEP).
\end{enumerate}

Foton se může také nejdříve rozptýlit mimo aktivní část detektoru a až poté být detekován. Energetické spektrum je opět omezeno minimální energií rozptýleného elektronu a geometrií detektoru. Proto ve spektru pozorujeme \emph{hranu zpětného odrazu}.

Také se může stát, že elektron-pozitronový pár vznikne mimo aktivní oblast detektoru. Jeden z fotonů potom většinou dolétne do aktivní části a my pozorujeme tzv. \emph{anihilační pík} na \kev{511}.



Statistickou nejistotu energií píků určujeme následujícím způsobem. Předpokládáme, že se jednotlivé události řídí normálním rozdělením. Pokud změříme $N$ událostí (net count) tak naměřené rozdělení bude mít standardní odchylku
\begin{equation}
\sigma_N=\frac{\text{FWHM}}{2\sqrt{2\log2}} \,,
\end{equation}
kde FWHM je šířka píku v polovině maxima, kterou nafituje program.
Nejistota určení středu rozdělení je potom
\begin{equation}
\sigma=\frac{\sigma_N}{\sqrt{N}}
\end{equation}