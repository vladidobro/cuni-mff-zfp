\section*{Diskuze}


Funkci $f(T)$ jsme určili až na škálovací faktor způsobený nepřesným $r$. Přesto závislost přibližně odpovídá teoretické závislosti \eqref{teoreticka}, což napovídá, že jsme vzdálenost $r$ odhadli správně.

V numerické derivaci \eqref{dTdP} odčítáme ve jmenovateli blízká čísla, což způsobuje chybu, pokud jsme tlak nezměřili přesně.

Tlak vzduchu byl $P_0=\SI{960}{\hecto\pascal}$, což bylo způsobeno jinou teplotou a vlhkostí než je normální ($P_0=\SI{1013}{\hecto\pascal}$). Proto jsme i specifické ionizační ztráty $f$ vztahovali ve vzorci \ref{f} k tlaku \SI{960}{\hecto\pascal} a dostali jsme funkci, jejíž teoretický tvar se může lišit od \eqref{teoreticka}.

Píky \Pudev a \Puosm se poměrně dobře shodují s tabelovanými hodnotami, jsou ale nižší, což bylo pravděpodobně způsobeno nedokonalým vakuem nebo kalibrací.

Ke kalibraci jsme použili pouze dva body (jeden pík a nulový kanál) a omezili jsme se tedy jen na lineární závislost čísla kanálu a energie. Nejsme tedy schopni posoudit kvalitu kalibrace.