\section*{Teoretická část}
Absolutní aktivita vzorku $A$ je celkový počet částic, který ze vzorku vyletí za jednotku času.
Pokud naměříme ve spektrometru aktivitu $a$, pak platí
\begin{equation}\label{aktivita}
A=a \frac{4\pi}{\Omega} = a \frac{r^24\pi}{S_v} = a \frac{2r}{r-\sqrt{r^2-S/\pi}}\,,
\end{equation}
kde $\Omega$ je pokrytý prostorový úhel, $r$ je vzdálenost terčíku od vzorku a $S_v$ je povrch pokrytého vrchlíku, který jsme určili z povrchu kruhového terčíku $S$
\begin{equation}
S_v=2\pi r (r-\sqrt{r^2-S/\pi})
\end{equation}


Předpokládáme, že hustota vzduchu je přímo úměrná tlaku. Potom ionizační ztráty při tlaku $P$ na vzdálenosti $r$ jsou stejné jako ionizační ztráty při atmosférickém tlaku $P_0$ a na vzdálenosti
\begin{equation}
x=r\frac{P}{P_0} \,.
\end{equation}
Specifické ionizační ztráty definujeme \cite{skripta}
\begin{equation} \label{f}
f(T):=-\frac{dT}{dx}=-\frac{P_0}{r}\frac{dT}{dP} \,.
\end{equation}
Derivaci $dT/dP$ budeme počítat numericky jako rozdíl dvou vedlejších bodů
\begin{equation} \label{dTdP}
\frac{dT}{dP}=\frac{T_{i+1}-T_i}{P_{i+1}-P_{i}} \,.
\end{equation}
Zdroj \cite{skripta} udává (po derivaci doletu částic $R$) teoretickou závislost
\begin{equation} \label{teoreticka}
f(T)=\frac{2}{3} \frac{1}{\xi \sqrt{T}} \,,
\end{equation}
kde $\xi=\SI{0.31}{\cm\MeV}^{-\frac{3}{2}}$ a $T$ v rozmezí \num{4}--\SI{7}{\MeV}.


Standardní nejistotu středu píku $\sigma_T$ určíme jako
\begin{equation}\label{chyba}
\sigma_T=\frac{\text{FWHM}}{2\sqrt{2\log 2}\sqrt{N}} \,,
\end{equation}
kde $N$ je celkový výtěžek náležící píku (\emph{net count}) a FWHM je pološířka píku.


Pokud máme vzorek dvou radioaktivních izotopů, u kterých známe poločasy rozpadu $T_{1/2}$, můžeme ze změřených aktivit určit jejich relativní molární podíl
\begin{equation} \label{Pu}
\frac{N(1)}{N(2)}=\frac{A(1) T_{1/2}(1)}{A(2) T_{1/2}(2)} \,,
\end{equation}
kde $A$ jsou aktivity. Pokud měříme stejný čas, je podíl aktivit rovný podílu výtěžků. Z jejich poměru už můžeme snadno určit relativní zastoupení
\begin{equation} 
\eta(1)=\frac{1}{1+\frac{N(2)}{N(1)}} \,, \qquad \qquad \eta(2)=\frac{1}{1+\frac{N(1)}{N(2)}} \,.
\end{equation}