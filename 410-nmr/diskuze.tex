\section*{Diskuze}

Závislosti na grafech \ref{g:t1} a \ref{g:tau} se velmi dobře shodují s teoretickými.
Naopak závislost na grafu \ref{g:t2} neodpovídá příliš přesně exponenciálnímu poklesu u vyšších časů.

Nejistotu u časů $T_1$ a $T_2$ považujeme za podhodnocenou, pokud například závislost na grafu \ref{g:t2} fitujeme (lineárně) až po zlogaritmování (což odpovídá přiřazení váhy bodům), dostaneme hodnotu \SI{1600}{\us}. Proto jsme v závěru nejistotu patřičně zvětšili a nejistota uvedená v části \emph{Výsledky měření} je pouze statistická chyba fitu.

Největší chybu metody způsobuje pravděpodobně nehomogenita pole, která je řádově vyšší než je u podobných přístrojů běžné.

Intenzita šumu klesá úměrně $1/\sqrt{N}$. Pro $N=200$ už je šum poměrně zanedbatelný.