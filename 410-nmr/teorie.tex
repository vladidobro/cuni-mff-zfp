\section*{Teoretická část}
Proton má magnetický moment
\begin{equation}
\bm{\mu}=\gamma\bm{I}\,,
\end{equation}
kde $\gamma$ je gyromagnetický poměr a $\bm{I}$ je jeho celkový moment hybnosti.
Pokud proton umístíme do magnetického pole velikosti $B_0$ ve směru osy $z$, bude magnetický moment protonu vykonávat \emph{Larmorovu precesi} s Larmorovvou frekvencí \cite{skripta}
\begin{equation} \label{e:fL}
f_L=\frac{\gamma B_0}{2\pi} \,.
\end{equation}

Označíme $\bm{M}$ jádernou magnetizaci.
V látce bude atom interagovat i s okolními atomy a jeho magnetický moment se bude stáčet do směru vnějšího magnetického pole. Pokud jsou magnetické momenty před časem $t=0$ orientované náhodně (žádné vnější pole, $\bm{M}=0$), a v čase $t=0$ zapneme magnetické pole, bude časový vývoj magnetizace (složky $z$) dán \cite{skripta}
\begin{equation} \label{e:mz}
M_z(t)=M_0 (1-\exp(-t/T_1)) \,,
\end{equation}
kde $M_0$ je rovnovážná magnetizace v daném poli a $T_1$ je tzv. spin-mřížková relaxační doba.

Pokud v čase $t=0$ byla naopak příčná složka $M_t$ rovna hodnotě $M_{t0}$, pak pro její vývoj v čase platí \cite{skripta}
\begin{equation} \label{e:mt}
M_t(t)=M_{t0} \exp(-t/T_2) \,,
\end{equation}
kde $T_2$ je tzv. spin-spinová relaxační doba. Příčná složka bude zanikat kvůli jak kvůli stáčení momentů do směru pole, tak kvůli nepatrně rozdílným Larmorovým frekvencím jednotlivých jader (nehomogenní pole).

Počáteční magnetizaci vybudíme polem $\bm{B_0}$. Za trigrovací dobu $T_0$ pustíme krátký harmonickým pulz ve směru osy $x$ s amplitudou $B_1$, frekvencí $f_L$ a délkou $\tau$. Tím otočíme magnetizaci okolo osy $x$ o úhel \cite{skripta}
\begin{equation} \label{e:uhel}
\varphi=\omega_1\tau=\gamma B_1 \tau \,.
\end{equation}
Volíme takové $\tau$, aby $\varphi=\pi/2$ ($\pi/2$-pulz). Magnetický moment je potom ve směru osy $y$.

Magnetizace se poté stáčí do směru pole $\bm{B_0}$ a my pozorujeme tzv. signál volné precese (FID), který je Fourierovým obrazem spektra NMR.

Po určité době dojde k utlumení FID signálu vlivem nehomogenního pole dojde k rozfázování jednotlivých momentů s různou $f_L$. Pokud nyní (po době $t_{12})$ od prvního $\pi/2$-pulzu) pustíme $\pi$-pulz, dojde po době $t_{12}$ k jejich opětovnému sfázování. Tomuto jevu říkáme spinové echo.

Standardní odchylka šumového napětí je
\begin{equation}
\sigma^2_{\bar{u^n}}=\sqrt{\frac{1}{L-1} \sum_{i=1}^{L} (\bar{u^n_i})^2  } \,,
\end{equation}
kde $\bar{u^n_i}$ je hodnota šumu v jednotlivých kanálech a $L$ je počet použitých kanálů. Pokud $u^n_i $ bude aritmetický průměr $N$ náhodných veličin se standardní odchylkou $\sigma_{u^n}$, pak bude díky centrální limitní větě platit \cite{skripta}
\begin{equation}
\sigma^2_{\bar{u^n}} \approx \frac{\sigma_{u^n}}{\sqrt{N}} \,.
\end{equation}