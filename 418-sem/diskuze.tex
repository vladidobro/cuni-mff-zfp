\section*{Diskuze}
Do nejistoty $\pm \SI{1}{zrno}$ každého kruhu je zahrnuta i případná systematická chyba, které jsme se s velkou pravděpodobností mohli dopustit. Proto je konečná nejistota $\bar{d}$ vysoká i přesto, že statistický soubor byl poměrně rozsáhlý.

Stejně tak při určování poměru fází v pájce. Shoda u obou snímků neznamená potvrzení naměřených hodnot, ale pouze to, že jsme v naší metodě konzistentní.