\section*{Teoretická část}
Skenovací elektronový mikroskop je přístroj, který pomocí fokusovaného svazku primárních elektronů postupně vytváří zvětšený obraz vzorku bod po bodu. Po dopadu na vzorek může nastat několik druhů událostí, jejichž detekce nám může poskytnout informaci o povrchu vzorku, my se zaměříme na dva z nich.

Elektron se může pružně odrazit zpět (BSE). Intenzita zpětně odražených elektronů je mimo jiné závislá na protonovém čísle (takže nám umožňuje pozorovat chemické složení vzorku) a na orientaci krystalografických rovin (tzv. channelling).

Elektron také může z atomů vzorku vyrazit sekundární elektron (SE). Tímto způsobem můžeme pozorovat topografii vzorku.

Pokud máme mikroskopický snímek zrnitého vzorku, můžeme z něj určit střední velikost zrna kruhovou metodou \cite{skripta}. Narýsujeme kružnici o průměru $D$ a spočítáme počet zrn, které protne. Tento počet označíme $n$. Potom střední velikost zrn $d$ je dána vztahem \cite{skripta}
\begin{equation} \label{e:kruhy}
d=\frac{3\pi}{2} \frac{D}{n} \,,
\end{equation}
kde $\pi$ je Ludolfovo číslo.