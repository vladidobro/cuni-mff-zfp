\section*{Teoretická část}
Skenovací elektronový mikroskop používá ke studiu vzorku zaostřený svazek elektronů.
V detektoru Augerových a sekundárních elektronů (SE) vidíme topografický povrch vzorku. Sekundární elektrony jsou elektrony vyražené ze vzorku a od primárních elektronů (PE) je poznáme tak, že mají mnohem menší energii (do cca \SI{50}{\eV}).
V detektoru zpětně odražených vzorků (BSE) vidíme Z-kontrast (atomové číslo), tedy kontrast mezi oblastmi prvků s velmi odlišným atomovým číslem. Vyšší Z znamená vyšší intenzitu BSE (tedy světlejší barvu na snímku). BSE mají energii srovnatelnou s energií PE.

Po dopadu PE na vzorek vzniká RTG záření dvojího druhu. Za prvé je to bzdné záření se spojitým spektrem, které budeme v našem experimentu filtrovat a nijak ho nevyužijeme, a za druhé je to charakteristické záření přítomných prvků s diskrétním spektrem. Každý prvek má charakteristické spektrum, které odpovídá energetickým přechodům při deexcitaci. Podle tohoto spektra dokážeme prvek ve vzorku identifikovat a určit jeho koncentraci.

Při kvantitativní analýze RTG spektra musíme brát v úvahu různé jevy, které se projeví na intenzitě spektrálních čar (korekce na atomové číslo, absorbce, fluorescence) \cite{skripta}. Naštěstí software dodávaný s mikroskopem tyto korekce provádí za nás.