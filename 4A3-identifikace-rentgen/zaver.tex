\section*{Závěr}
Provedli jsme energetickou kalibraci pomocí tří peaků v $\gamma$-spektru $^{241}$Am: \SI{13.9}{\keV}, \SI{26.3}{\keV} a \SI{59.5}{\keV}.

Určili jsme materiál 7 vzorků, viz tabulka \ref{t:merenivzorky}.

Změřili jsme relativní zastoupení prvků ve vzorcích 5
\begin{equation*}
w_{Cu}^5=\SI{28(5)}{\percent} \,,\qquad \qquad w_{Ag}^5=\SI{72(5)}{\percent}
\end{equation*}
a 13
\begin{equation*}
w_{Pb}^{13}=\SI{74(5)}{\percent} \,,\qquad \qquad w_{Sn}^{13}=\SI{26(5)}{\percent} \,.
\end{equation*}

Sestavili jsme závislost výtěžku na protonovém čísle, viz graf \ref{g:vytezek}. Závislost pro nízká Z rychle klesá.

Určili jsme radioaktivní vzorek $^{133}$Ba, který se měnil na $^{133}$Cs záchytem elektronu.